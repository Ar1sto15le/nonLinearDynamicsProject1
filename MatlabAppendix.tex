\pagebreak{} % so we do not overlap with bibliography
\section*{MATLAB Appendix: \\ Scripts for Generating Bifurcation Diagrams and Iterative Numerics}

\begin{verbatim}
function bloodModelVaryac0(steps,b,r,s)

% Pressing "Run" button uses the below values as the inputs
%   for the function.
if(nargin==0)
    steps=200;
    b=1.1*10.^6;
    r=8;
    s=16;
end

ICs=10; % Number of initial conditions
c0=linspace(0,1,ICs); % Initial conditions tested between 0 and 1.5
runs=70; % Number of different `a' values to test
a=linspace(.6,.75,runs).'; % Linearly spaced vector of `a' values
cn=zeros([1, steps]); 

%Starts clock on main loop
t=cputime;

% Loop through initial condition values
for k=1:ICs
    cn(1)=c0(k);    
    % Loop through the `a' values
    for j=1:runs    
        % Loop through the first (steps-50) steps
        for i=1:steps-50
        cn(i+1)=(1-a(j)).*cn(i)+b*cn(i).^r*exp(-s*cn(i));
    end
    
    % Plots new data on the same graph
    hold on
    
    % Loop through final steps while plotting bifurcation diagram.
    % Plotting just these steps ensures that we only use values that 
    %   are in the periodic orbits that the map converges to for a given
    %   `a'.
    for i=steps-50:steps
        cn(i+1)=(1-a(j)).*cn(i)+b*cn(i).^r*exp(-s*cn(i));
        plot(a(j), cn(i),`-', `MarkerSize',2);
    end
    end
end
% Determines how long the main loop takes to run
loopTime=cputime-t;
display(loopTime);

xlabel(` a ');
ylabel(` c_n for last 50 values')
title(`Bifurcation diagram found by varying ``a''');

end

---

function bloodModelr5(c0,steps,b,r,s)

% Pressing ``Run'' button uses the below values as the inputs
%   for the function.
if(nargin==0)
    c0=1; % Initial condition
    steps=100; % Number of iterations of the map
    b=1.1*10.^6; % Value of parameter given in part (b)
    r=4.9;         % "                                  "
    s=16;        % "                                  "
end

a=.2; % Value given in part (b)
% a=.3;  % Value given in part (c) 
cn=zeros([1,steps]); % Create a vector to hold value at each step
cn(1)=c0; % Initial condition
    
% Main loop to generate iterative map
for i=1:steps
    cn(i+1)=(1-a).*cn(i)+b*cn(i).^r*exp(-s*cn(i));
end

% Simple plot of result
figure
plot(cn(:));
end

---
function bloodModelSimple(c0,steps)

if(nargin==0)
    c0=linspace(0,1,50);
    steps=20;
end
a=.3;
b=1.1*10.^6;
s=16;
r=8;
cn=zeros([1 steps]);
xlabel(`iteration');
ylabel(`c_n');
title(`Iterative map with varying initial conditions');
for k=1:50
    cn(1)=c0(k);
    for j=1:steps
        cn(j+1)=(1-a).*cn(j)+b*cn(j).^r*exp(-s*cn(j));
    end
    hold on
    plot(cn,`-');
end
end

---
function bloodModelSinglea(c0,steps,b,r,s)

if(nargin==0)
    c0=1.5;
    steps=1000;
    b=1.1*10.^6;
    r=8;
    s=16;
end

runs=200;
a=linspace(0,1,runs).';
cn=zeros([1, steps]);
cn(1)=c0;

for j=1:runs
    
    for i=1:steps-100
        cn(i+1)=(1-a(j)).*cn(i)+b*cn(i).^r*exp(-s*cn(i));
    end
    hold on
    for i=steps-100:steps
        plot(a(j), cn(i),`-', `MarkerSize',2);
    end
end
end


---

function bloodModelVaryA(c0,steps,b,r,s)

% Pressing "Run" button uses the below values as the inputs
%   for the function.
if(nargin==0)
    c0=1.5;
    steps=300;
    b=1.1*10.^6;
    r=8;
    s=16;
end

runs=500; % Number of different `a' values to test
a=linspace(0,1,runs).'; % Linearly spaced vector of `a' values
cn=zeros([1, steps]); 
cn(1)=c0;

% Loop through the `a' values
for j=1:runs
    
    % Loop through the first (steps-50) steps
    for i=1:steps-50
        cn(i+1)=(1-a(j)).*cn(i)+b*cn(i).^r*exp(-s*cn(i));
    end
    
    % Plots new data on the same graph
    hold on
    
    % Loop through final steps while plotting bifurcation diagram.
    % Plotting just these steps ensures that we only use values that 
    %   are in the periodic orbits that the map converges to for a given
    %   `a'.
    for i=steps-50:steps
        cn(i+1)=(1-a(j)).*cn(i)+b*cn(i).^r*exp(-s*cn(i));
        plot(a(j), cn(i),`-', `MarkerSize',2);
    end
end

xlabel(` a ');
ylabel(` c_n for last 100 n values')
title(`Bifurcation diagram found by varying "a"');

end


---

function bloodModelVaryac0(steps,b,r,s)

% Pressing "Run" button uses the below values as the inputs
%   for the function.
if(nargin==0)
    steps=200;
    b=1.1*10.^6;
    r=8;
    s=16;
end

ICs=10; % Number of initial conditions
c0=linspace(0,1,ICs); % Initial conditions tested between 0 and 1.5
runs=70; % Number of different `a' values to test
a=linspace(0,1,runs).'; % Linearly spaced vector of `a' values
cn=zeros([1, steps]); 

%Starts clock on main loop
t=cputime;

% Loop through initial condition values
for k=1:ICs
    cn(1)=c0(k);    
    % Loop through the `a' values
    for j=1:runs    
        % Loop through the first (steps-50) steps
        for i=1:steps-50
        cn(i+1)=(1-a(j)).*cn(i)+b*cn(i).^r*exp(-s*cn(i));
    end
    
    % Plots new data on the same graph
    hold on
    
    % Loop through final steps while plotting bifurcation diagram.
    % Plotting just these steps ensures that we only use values that 
    %   are in the periodic orbits that the map converges to for a given
    %   `a'.
    for i=steps-50:steps
        cn(i+1)=(1-a(j)).*cn(i)+b*cn(i).^r*exp(-s*cn(i));
        plot(a(j), cn(i),`-', `MarkerSize',2);
    end
    end
end
% Determines how long the main loop takes to run
loopTime=cputime-t;
display(loopTime);

xlabel(` a ');
ylabel(` c_n for last 50 values')
title(`Bifurcation diagram found by varying ``a''');

end

---

function bloodModelvaryr(c0,steps,b,s)

% Pressing "Run" button uses the below values as the inputs
%   for the function.
if(nargin==0)
    c0=.5; % Initial condition
    steps=300; % Number of iterations of the map
    b=1.1*10.^6; % Value of parameter given in part (b)
    s=16;        % "                                  "
end

rVals=300;
r=linspace(0,10,rVals);
% a=.2; % Value given in part (b)
a=.3;  % Value given in part (c) 
cn=zeros([1,steps]); % Create a vector to hold value at each step
cn(1)=c0; % Initial condition

t=cputime;
% Main loop to generate iterative map
for k=1:rVals
    
    for i=1:steps-100
        cn(i+1)=(1-a).*cn(i)+b*cn(i).^r(k)*exp(-s*cn(i));
    end
    hold on
    for i=steps-100:steps
        cn(i+1)=(1-a).*cn(i)+b*cn(i).^r(k)*exp(-s*cn(i));
        plot(r(k),cn(i),`-', `MarkerSize',2);
    end
end
loopTime=cputime-t;
display(loopTime);
xlabel(` r ');
ylabel(` c_n for last 50 values')
title(`Bifurcation diagram found by varying "r"');
end

---

function bloodModelVarys(c0,steps,b,r)

% Pressing "Run" button uses the below values as the inputs
%   for the function.
if(nargin==0)
    c0=.5; % Initial condition
    steps=200; % Number of iterations of the map
    
    b=1.1*10.^6; % Value of parameter given in part (b)
    r= 8;        % "                                  "
    % s=16;        % "                                  "
end

sVals=200;
s=linspace(8,20,sVals);
% a=.2; % Value given in part (b)
a=.3;  % Value given in part (c) 
cn=zeros([1,steps]); % Create a vector to hold value at each step
cn(1)=c0; % Initial condition

t=cputime;
% Main loop to generate iterative map
for k=1:sVals
    
    for i=1:steps-20
        cn(i+1)=(1-a).*cn(i)+b*cn(i).^r*exp(-s(k)*cn(i));
    end
    hold on
    for i=steps-20:steps
        cn(i+1)=(1-a).*cn(i)+b*cn(i).^r*exp(-s(k)*cn(i));
        plot(s(k),cn(i),`-', `MarkerSize',2);
    end
end
loopTime=cputime-t;
display(loopTime);
xlabel(` r ');
ylabel(` c_n for last 20 values')
title(`Bifurcation diagram found by varying "s"');
end

---

function c0findBifurcation
    format long
    upper=.2;
    lower=.1;
    steps=50;
    a=.3;
    b=1.1*10.^6;
    s=16;
    r=8;
    cn=zeros([1 steps]);
    c0=.15;
    tol = 1*10.^(-13);
    c0temp=1;
    while abs(c0temp-c0) > tol
        
        cn(1)=c0;
        for j=1:steps-1
            cn(j+1)=(1-a).*cn(j)+b*cn(j).^r*exp(-s*cn(j));
        end
        if cn(steps)< .5
            c0temp=c0;
            lower=c0;
            c0=(upper+c0)/2;
        else
            c0temp=c0;
            upper=c0;
            c0=(lower+c0)/2;
        end
    end
    display(c0);

\end{verbatim}
